% !TEX encoding = UTF-8 Unicode
\section{Démocratie athénienne réinterprétée}
\subsection{Modèle athénien : la démocratie représentative
  non-électorale.}
Nous apprenons à l'école que le système athénien était une forme de
démocratie <<directe>>, où les citoyens prenaient des décisions dans les
assemblées, sans intermédiaires, à l'échelle d'une assemblée. Nous
apprenons 
que même si ce genre de système pourrait fonctionner à petite échelle,
comme une réunion de ville de la Nouvelle Angleterre
\cite{bryan03}, cela serait impossible dans une grande
ville, et encore moins dans une nation.\footnote{%
  Certains partisans de la démocratie directe s'accrochent au
  référendum comme l'héritier le plus proche du modèle Athénien (bien
  que cela ne soit pas la délibération essentielle pour le modèle athénien sans
  représentant). Certains proposent des façons novatrices d'utiliser
  les télécommunications et l'Internet pour résoudre ce problème
  d'échelle. Mais comme  \cite{dahl70} l'a fait remarquer dans
  son livre concis \cite{dahl70} \cite{dahl70}, le nombre de
  participants serait tel que le temps nécessaire à l'organisation des
  débats les priverait de toute autre forme d'activité. Il en irait de
  même si seul un petit pourcentage d'entre eux participait (prise de
  parole ou écriture, lecture de ce qui a été écrit) \cite{dahl70}.%
} 
Nous sommes amenés à la conclusion qu'il n'y a
pratiquement rien de la démocratie athénienne que nous pouvons
utiliser aujourd'hui au-delà de sa valeur d'inspiration. Toutefois, un
examen attentif du fonctionnement de la démocratie athénienne, et de la
manière dont les Athéniens l'ont améliorée sur près de 200
ans\footnote{%
  Il est fréquent d'imaginer la démocratie athénienne telle
  qu'elle existait au temps de \textsc{Périclès} et sa célèbre oraison
  funèbre. Toutefois, les Athéniens ont continué à faire des
  changements.  \cite{wood05} écrit dans \cite{wood05},
  "Après une guerre civile ou un grand échec militaire, les Athéniens
  auraient ajusté leur système afin de mieux se conformer aux
  objectifs de la démocratie. Un exemple frappant en était la
  toute puissance de la législation en dehors de l'Assemblée au
  quatrième siècle, alors exprimée par un corps de représentation"
  \cite{wood05}.%
  }, révèle une histoire très différente. 

La démocratie athénienne --- en particulier la démocratie classique après
403 avant notre ère, telle qu'elle est décrite dans \cite{hans99}
\cite{hans99} \cite{hans99} était fondamentalement
représentative plutôt que directe. Jamais plus qu'une petite fraction
de l'ensemble des citoyens de sexe masculin d'Athènes ne se réunissait
pour voter. Le lieu de réunion de l'Assemblée populaire ne pouvait
contenir que 6 000 et plus tard peut être 8 000, sur un total estimé
de 30 000 à 60 000 citoyens éligibles. Ainsi, l'Assemblée du peuple
était un échantillon des \emph{demos}, mais les décisions qui y
étaient prises étaient traitées comme si la totalité des \emph{demos}
avait voté.\par  
Qui plus est, la plupart des décisions de gouvernance n'étaient pas
prises par l'Assemblée du peuple, mais par des petits groupes
représentatifs de citoyens. Ces représentants n'étaient pas élus. Ils 
étaeint choisis par tirage au sort.\par
L'Assemblée du peuple ne discutait généralement pas une question
jusqu'à ce qu'elle eut été examinée par le Conseil des 500 choisi au
hasard (\emph{boule}). Le chercheur classique \cite{ober07} a
mis en évidence que le conseil choisi par tirage au sort était
l'institution clé dans la démocratie grecque. Il peut même avoir eu
un rôle plus central dans la conception grecque de la démocratie, peut
être plus que l'Assemblée du peuple \cite{ober07}. Les cités
non démocratiques, comme Sparte, avaient des assemblées, mais
le calendrier politique\footnote{agenda en anglais}
était contrôlé par l'aristocratie. \`A Athènes, le Conseil des 500
fixait l'ordre du jour\footnote{agenda en anglais}, et préparait les
décrets et résolutions. Les groupes législatifs sélectionnés
aléatoirement (\emph{nomothetai}) de 1001 citoyens de 
plus de 30 ans, devaient approuver les nouvelles lois. Les tribunaux
populaires (\emph{dikasteria}), généralement 501, 1001 ou 1501
citoyens choisis par tirage au sort, pouvaient statuer sur l'Assemblée
populaire\footnote{%
  Le biographe et historien grec \textsc{Plutarque} a
  suggéré que la Cour populaire choisie au hasard avait donc
  l'autorité souveraine ultime, plutôt que l'Assemblée du
  peuple.%
}. Presque tous les magistrats qui ont mené l'entreprise
de gouvernance ont également été choisis par tirage au sort, le plus
souvent par groupes de 10 citoyens.\footnote{%
  Il y avait, bien sûr des exceptions. Prenant le concept de
  "fonctionnaire" au pied de la lettre, certains responsables gouvernementaux,
  tels que les approbateurs qui ont régné sur l'authenticité des
  pièces d'argent utilisées dans le marché public (\emph{Agora}) et au
  port (\emph{Piraeus}) devaient être de réels esclaves
  \cite{ober08}.%
}\par  
Seuls quelques postes de direction spécialisés, tels que les généraux
et les directeurs financiers, étaient pourvus par des élections à
l'Assemblée du peuple. Les Athéniens considéraient les élections comme
intrinsèquent aristocratiques, car seuls ceux qui avaient de l'argent
et le statut pouvaient gagner. Pour les Athéniens, le tirage au sort
était un élément essentiel de la démocratie. En fait, c'était l'opinion
générale parmi les théoriciens politiques d'\textsc{Aristote} à
\textsc{Montesquieu} et \textsc{Rousseau} \cite{manin97}.\par
La démocratie athénienne n'était pas fondée sur le principe que tous
les citoyens doivent participer à toutes les décisions. Cela aurait
été aussi impossible à Athènes que ça l'est aujourd'hui. Cependant, il
y a des principes importants et des pratiques de la démocratie
athénienne qui peuvent être appliqués aujourd'hui.\par
Le premier principe est l'\emph{isonomia} --- le droit égal de tous les
citoyens d'exercer leurs droits politiques. Grâce au tirage au sort,
tous les citoyens qui le souhaitaient avaient une chance égale et une
forte probabilité de servir dans la fonction publique. Ceci est
fondamentalement différent de la chance extrêmement inégale d'accéder
au bureau politique par l'élection.\par
Le deuxième principe est \emph{isegoria} --- tout citoyen a le droit
de parler à l'Assemblée du peuple et de faire des propositions. Peu de
citoyens n'ont jamais réellement parlé à l'Assemblée, mais le droit de
tout citoyen à ajouter de nouvelles informations ou arguments était
considéré comme fondamental.\par 
Ceci est différent du fait d'avoir un droit individuel de vote
comptabilisé. Le vote d'un seul individu dans l'Assemblée du peuple à
Athènes, comme dans les élections aujourd'hui, avait peu
d'importance. Le vote d'un seul individu ne peut déterminer le
résultat d'une élection. En effet, les votes sur la plupart des
questions devant l'Assemblée populaire n'ont jamais été effectivement
comptés.\footnote{%
  Les votes par scrutin secret à l'aide de disques de
  vote déposés dans les urnes de bronze \emph{étaient} comptés dans la
  Cour populaire choisie au hasard et dans les groupes législatifs, ou
  dans des cas uniques dans l'Assemblée du peuple comme pour le
  bannissement ou lorsque le quorum était nécessaire.%
} Au lieu de
cela, neuf citoyens étaient choisis au hasard simplement à main
levée.\par 
\cite{ober08} a fait valoir que la capacité institutionnelle de la
démocratie athénienne à exploiter la propagation de la connaissance
latente et diffuse dans la population est un facteur déterminant qui lui
permettait de s'épanouir \cite{ober08}. La vraie signification de 
\emph{isegoria} est l'occasion de tout citoyen de donner des
informations, plutôt qu'un simple vote. Contrairement à un vote, un
seul élément d'information a le potentiel déterminant pour influencer la
décision finale. L'\emph{isegoria} n'était pas seulement un droit
individuel, mais aussi un avantage pour la communauté. La
polis\footnote{%
  Pour une explication de ce terme consulter
  \url{https://fr.wikipedia.org/wiki/Polis}%
} ne souffrirait pas de l'absence de vote d'un individu, mais pourrait
perdre beaucoup si un citoyen avec des informations cruciales 
ou des arguments pertinents se voyait refuser le droit de
participer. L'Assemblée du peuple prendrait en conséquence une
mauvaise décision. L'\emph{isegoria} protège ces <<actes de parole>>
plutôt que le droit de vote.\par

\subsection{La question de l'échelle}
Comprendre que la démocratie athénienne était représentative --- mais
sous une forme très différente de celle que nous connaissons
aujourd'hui --- nous conduit à une autre idée importante. La plupart des
études modernes sur la démocratie rejettent le système athénien comme
inapplicable aux Etats-nations modernes (ou même des villes) en raison
de la question d'échelle. Certains affirment que la démocratie n'est
tout simplement pas possible à grande échelle, tandis que d'autres redéfinissent
tout simplement  la démocratie en remplaçant les systèmes
électoraux modernes par la définition originale. En fait, les
Athéniens avaient \emph{résolu} le problème d'échelle --- le coeur du
problème qui avait entravé des théoriciens de la démocratie et les
hommes politiques pour les dernières centaines d'années.\par
Une population de 30 000 citoyens peut paraître petite selon les
normes modernes, mais elle est beaucoup trop grande pour la démocratie
non représentative et <<participative>> telle que nous la pensons
aujourd'hui. Les Athéniens ont inventé un système de gouvernement qui
fonctionnait à une plus grande échelle que celle d'une assemblée, dans lequel
les citoyens gouvernaient à travers des institutions
\emph{représentatives}. Cela s'appelait "démocratie".\par
Même l'Assemblée du peuple, comme il est indiqué ci-dessus, avait un
caractère représentatif. Avec la compréhension moderne de la
probabilité et de l'échantillonnage scientifique, nous savons qu'un
échantillon représentatif n'a pas besoin de continuer à grandir
proportionnellement à la croissance de la population
échantillonnée. Un échantillon de 6 000 citoyens (typique de
l'Assemblée du peuple) pourrait représenter fidèlement une population
de 300 000 000 aussi bien que 30 000. \par
Certains contestent mon affirmation que la démocratie athénienne était
représentative. Certains ont soutenu que le tirage au sort était
simplement un moyen efficace d'atteindre le principe de "gouverner
et être gouverné à son tour" par alternance
\cite{manin97} ou peut-être s'agissait-il de laisser le choix aux dieux. Cet
argument affirme que les titulaires de charges n'étaient pas
considérés comme des représentants des communautés, des classes ou des
tribus d'où ils venaient \cite{dowlen08}. Certaines preuves du
contraire viennent du fait que chacune des 139 unités géographiques de
l'Attique (villages et quartiers environnants d'Athènes, connus sous
le nom de \emph{demes}) avaeint droit à un certain nombre de sièges au
Conseil des 500, en proportion de leur population \cite{hans99}.\par 
On peut aussi faire valoir que les Athéniens, en dépit de leurs
progrès étonnants en mathématiques, ne connaissaient pas les
probabilités, et n'avaient pas de "théorie de la représentation"
\cite{pitkin67}. Mais ces corps ne fonctionnaient efficacement
qu'en tant que représentants de l'ensemble des citoyens. Comme le
théoricien du tirage au sort \cite{sutherland08} l'a noté, tout
cuisinier athénien savait qu'en goûtant une cuillerée, on obtenait une bonne
idée de la qualité de la soupe dans son ensemble
\cite{sutherland08}. Nous pouvons également noter que les 
Athéniens n'avaient pas encore de "théorie de la gravité", et allaient
de l'avant et utilisaient la gravité dans des tâches quotidiennes de
façons différentes.\par 
Il y avait deux concepts qui étaient au coeur de la démocratie
athénienne, et qui peuvent être profondément utiles aujourd'hui. La
sélection aléatoire (tirage au sort) en était un. L'autre était de
répartir les pouvoirs politiques parmi de \emph{multiples} corps
choisis au hasard avec des caractérisitiques différentes.\par

\subsection{Les  corps multiples}
Dans la démocratie athénienne, la plupart des processus de décision
étaient répartis entre des corps distincts. Le Conseil des 500
fixait l'ordre du jour, et préparait les décrets préliminaires et les
résolutions à envisager par l'Assemblée, mais ne pouvait pas adopter
les lois. Le passage d'un décret par l'Assemblée du peuple pouvait
être renversé par un tribunal populaire, mais ces tribunaux ne
pouvaient pas adopter les lois elles-mêmes.\par
Suite à la codification de 402 avant notre ère, l'Assemblée du peuple
ne pouvait plus passer des lois. Au lieu de cela l'Assemblée pouvait
seulement entamer le processus en appelant à la création de groupes
législatifs choisis au hasard à usages simples, qui devaient passer de
nouvelles lois. Comme le note \cite{hans99}, c'était une réforme
bénéfique, car "le double examen d'une proposition a permis la
possibilité d'aboutir à une meilleure décision." Il a également donné
"du répit pour surmonter les effets de la psychose de masse qu'un
orateur habile aurait pu attiser dans une situation très tendue"
\cite{hans99}.\par 
La séparation athénienne des pouvoirs entre plusieurs organismes
choisis au hasard et les participants eux-mêmes sélectionnés de
l'Assemblée du peuple atteint trois objectifs importants que nos
Parlements modernes élus n'atteignent pas : 
\begin{enumerate}
\item Les corps législatifs étaient une représentation relativement
descriptive de l'ensemble des citoyens ;
\item Ils étaient très résistants à la corruption et à une concentration
excessive du pouvoir politique ;
\item La possibilité de participer --- et de prendre des décisions --- a
largement été étendue dans la population concernée.
\end{enumerate}
Dans la section suivante, je vais discuter de propositions
contemporaines pour donner au tirage au sort un rôle plus large au
sein du gouvernement, en particulier dans la branche législative. Je
vais faire valoir qu'elles ne répondent pas aux trois objectifs
ci-dessus, principalement parce qu'elles proposent chacune seulement un
usage exclusif d'un corps choisi au hasard. Ensuite, je vais vous
présenter un modèle de processus législatif en utilisant
plusieurs organismes désignés. Les éléments de cette conception
peuvent être appliqués à différents niveaux de gouvernement (local,
régional, national, ou international), et avec différents degrés de
saturation. Il peut être utilisé sur une base unique pour une seule
loi, comme c'est le cas dans l'Assemblée des citoyens de la
Colombie-Britannique, ou aussi pour les citoyens tirés au hasard en
groupes chargés de l'élaboration du budget de la banlieue de Sydney
Canada Bay en Australie. Il peut être appliqué sur une base continue à
un domaine de la législation. Ces corps pourraient être particulièrement
intéressants pour les législateurs élus qui souhaitent éviter des
décisions impopulaires, comme avec la Commission militaire de base
de réalignement et de fermeture établie par le Congrès. Cette
conception peut être appliquée pour remplacer une chambre dans un
système bicaméral, tout en conservant une chambre élue. Il pourrait
même être appliqué sans tenir compte des législateurs élus.\par















