\section*{Introduction}

Un certain nombre de chercheurs ont proposé des réformes démocratiques
qui utilisent la sélection aléatoire (tirage au sort) pour former des
corps délibérants (mini publics). Les défenseurs du tirage au sort
se tournent souvent vers la démocratie athénienne classique afin d'y
trouver une source d’inspiration. Cependant, la plupart des théoriciens
politiques rejettent la démocratie athénienne argumentant qu’elle ne
peut pas apporter de leçons pratiques pour les états-nations modernes,
en raison de la question d’échelle. Je ferai valoir que certaines
pratiques et principes démocratiques athéniens peuvent relever le défi
de l'échelle, et qu'ils peuvent être utilisés pour concevoir des
systèmes législatifs qui sont supérieurs à tout système législatif
utilisé par les démocraties représentatives modernes. Mon intention
est de ne pas idéaliser la démocratie athénienne --- il s'agissait d'une
société qui a tenu les esclaves, les femmes et les métèques exclus de
la citoyenneté, et a créé un empire par la conquête d'autres
cités-états. Cependant, je ferai valoir que certains aspects de la
démocratie athénienne contiennent de précieuses leçons pour la réforme
des gouvernements actuels.\par
Les trois pratiques fondamentales athéniennes étaient : 
\begin{enumerate}
\item Le choix des législateurs et des autres corps délibérants par tirage
au sort plutôt que par élection.
\item La répartion des tâches législatives entre plusieurs corps, chacun
ayant ses caractéristiques propres.
\item L'utilisation à la fois de corps ponctuels et de corps en 
cours de mandat à durée déterminée dans les processus de prise de décision.
\end{enumerate}
Cette structure permet une performance optimale : les tâches
législatives correspondent aux caractéristiques propres à chaque type
de corps, tout en minimisant les risques de thésaurisation et de
corruption par le pouvoir.\par  
Les principes démocratiques athéniens fondamentaux qui sous-tendent le
système que je vais proposer sont :
\begin{itemize}
\item le principe d'égalité politique (\emph{isonomia}\footnote{Pour
    plus de détails consulter : \url{https://fr.wikipedia.org/wiki/Isonomie}})
\item le droit de s'exprimer et de participer (\emph{isegoria})
\item la croyance en la capacité d'un échantillon de personnes à
délibérer, à argumenter, et à prendre des décisions raisonnables
\end{itemize}

