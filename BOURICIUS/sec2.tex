% !TEX encoding = UTF-8 Unicode
\section{Propositions modernes de tirage au sort et leurs problèmes}  
\subsection{Propositions modernes de tirage au sort}
Un certain nombre de défenseurs modernes de la démocratie
délibérative ont proposé une variété de systèmes de tirage au
sort. Bon nombre de ces propositions ont concerné des corps
ponctuels ou consultatifs (\autocite{dien95}; \autocite{dahl89};
\autocite{crosby86}; \autocite{fishkin09}), tandis que d'autres ont
proposé des systèmes pour institutionnaliser les corps délibérants
choisis au hasard, souvent avec autorité de prise de décision réelle
(\autocite{burn85}; \autocite{call85}; \autocite{gast00};
\autocite{cars99}, \autocite{cars05}; \autocite{goll07};
\autocite{leib04}; \autocite{oleary06}; \autocite{sutherland08};
\autocite{zak10}). Tous ces plans visent à renforcer 
une véritable délibération, augmenter la représentation
descriptive,\footnote{Le terme "représentation descriptive" désigne
  des représentants qui "ressemblent" à ceux qu'ils représentent. Je
  ne veux pas dire simplement la race ou le sexe, mais aussi des
  intérêts, des expériences de vie et des croyances comme celles qui
  sont représentées.} et réduire la corruption. Ils cherchent
également à surmonter l'ignorance rationnelle de l'électeur , tout en
reconnaissant que, dans la plupart des élections, la chance que son
vote changera réellement le résultat est si faible qu'il est
irrationnel de passer du temps ou de l'effort à en apprendre plus sur
les candidats ou les questions.\footnote{D'autres aspects
  psychologiques négatifs du vote aux élections de masse
  (\autocite{caplan07}; \autocite{westen07}) sont également préoccupants, et
  peuvent être observés lors de délibérations en petits groupes.}\par 
La plupart de ces auteurs ont proposé des systèmes comprenant un seul
corps choisi au hasard.\footnote{Je me réfère à des organismes
complètement séparés, plutôt qu'à un comité (extrait) d'un corps plus
grand. L'Assemblée citoyenne de la Colombie-Britannique, par exemple,
a utilisé un processus de sélection aléatoire des comités au sein
d'un ensemble plus vaste.(\autocite{herath07})} Il y a des
exceptions. \autocite{gast12} ont proposé un modèle de
tirage au sort multi-corps utilisant jusqu'à cinq corps distincts,
et aussi d'ajouter des éléments de la délibération démocratique au 
processus actuel d'initiative référendaire dans divers \'Etats
américains (\autocite{gast12}). \autocite{calvert12} préconisent
plusieurs mini-publics pour une seule question (\autocite{calvert12}), et
Lyn \textsc{Carson} et Janette \textsc{Hartz-Karp} discutent du concept
de relier plus d'une méthode de délibération dans un système combiné,
notant que le délibératif démocrate "Luigi \textsc{Bobbio}, à une
occasion, a suggéré la possibilité de convoquer deux [jury citoyens]
sur le même sujet, chaque jury ayant une composition différente --- un
jury pour les militants et un jury pour les citoyens choisis au hasard pour
évaluer leurs résultats respectifs" (\autocite{cars06}).\par
Le livre de référence de  \autocite{burn85}, \autocite{burn85},
décrit un système qu'il appelle "demarchy", composé entièrement de
corps choisis au hasard divisés en domaines fonctionnels
(\autocite{burn85}). Il propose également des "corps de niveau 
supérieur" disctincts qui superviseront et fourniront un cadre
juridique pour les organismes de prise de décision pour régler les
différends. Ce concept de méta-corps législatif, qui ne déclenche ou
ne prend des décision politique, est l'un des tremplins pour le
modèle que je propose.\par
\subsection{Motivation à participer}
Tout système proposé qui nécessite une augmentation de la
participation citoyenne, doit répondre à la 
question de savoir s'il y aurait une motivation suffisante dans de
larges pans de la population pour participer. Après tout, seule une
minorité de citoyens est prête à voter dans la plupart des élections
américaines, ce qui nécessite peu de temps et un effort relativement
minime. Les communautés de Nouvelle-Angleterre qui ont encore des
réunions de la ville ne voient aussi qu'une fraction de leurs citoyens
présents. Les usages en concurrence pour le temps personnel dans la
société moderne, et le "manque d'attractivité" de la politique pour la
plupart des gens, soulèvent de sérieuses questions quant à la
viabilité d'une telle entreprise démocratique, en particulier celle
centrée sur la délibération et une large participation
(\autocite{warren96}). Même dans l'Athènes classique, la démocratie était
une activité qui concernait uniquement la (relativement importante) partie des
citoyens qui avait choisi de participer. L'objectif de ce plan est
d'aller vers un <<meilleur Athènes>>, et d'inclure la population en
général, plutôt que simplement ceux qui sont désireux de participer.\par
L'hypothèse (vérifiable) que je fais ici, c'est que la plupart des
citoyens seraient prêts à participer pour une période de temps
définie, avec une compensation appropriée, dans un processus dans
lequel ils croieraient que leur action importerait vraiment (à la
différence des élections de masse). Ce processus démocratique n'aurait
presque aucun rapport avec la "politique" que nous connaissons
aujourd'hui. Alors que les niveaux de satisfaction élevés 
rapportés par les participants dans les différents processus de
délibération, tels que l'Assemblée des citoyens de Colombie
Britannique, ou les conférences de consensus danoises (\autocite{fischer09}),
peuvent être trompeurs (puisque ceux-ci ont été choisis au hasard parmi
ceux qui ont déjà dit qu'ils étaient intéressés), être l'un de ceux
"sélectionné" a le potentiel pour surmonter le problème de
l'"ignorance rationnelle" des élections de masse. Tout comme des jurés
dans les systèmes de justice peuvent se plaindre de la nuisance due à leur
service, ils prennent presque universellement leur travail au
sérieux. En effet, de nombreux jurés s'en vont avec un sens aigu de la
citoyenneté (\autocite{matthews04}). Le système décrit ci-dessous cherche
également à accueillir différents niveaux de volonté de consacrer du
temps personnel à l'auto-gouvernance. La plus grande partie des
participants serait mandatée un temps très limité --- par exemple, pas
plus d'une semaine.\par
\subsection{Cinq dilemnes de conception de tirage au sort}
Toutes les propositions de tirage au sort d'un seul corps doivent faire
face à cinq dilemmes --- cinq paires d'objectifs opposés --- qui ne
peuvent être conciliés avec un seul type d'corps.
\begin{enumerate}
\item Il y a un conflit entre la maximisation de la représentativité
descriptive, par rapport à la maximisation de l'intérêt et
l'engagement des membres d'un corps délibérant. Dans
\autocite{leib04}, \autocite{leib04} cherche à maximiser la
représentativité descriptive et éviter le biais de "distorsion
participative" en insistant sur le service obligatoire comme dans un
jury ou projet (\cite{leib04}). D'autres mettent l'accent sur la
garanite d'intérêt et de motivation. Leurs créations ont tendance à
aller vers le bénévolat, ou une loterie de la volonté.
\item Il y a un conflit entre l'augmentation de la participation et de la
résistance à la corruption par le biais d'un mandat courte durée, par
rapport à la maximisation de l'expertise ou de la familiarité des
participants avec les questions examinées par plusieurs fois ou
répétées.
\item Il y a le conflit entre donner à chaque citoyen le droit à la
parole (auto-sélection) --- offrant des articles de l'agenda,
des informations et des arguments pour le processus délibératif
(\emph{isegoria}), contre le danger que l'auto-sélection des personnes
les plus motivées à parler favorisera la domination par des intérêts
spéciaux et les résultats de détournement de l'intérêt commun.
\item Il y a un conflit entre le désir d'un corps diversifié qui
s'engage dans la résolution de problèmes par la délibération active,
contre l'évaluation personnelle indépendante (la "délibération
privée") qui exploite la "sagesse des foules" et évite les cascades
d'information, qui peut fermer la connaissance privée. Il y a des
recherches convaincantes montrant la valeur de la diversité cognitive
pour la résolution de problèmes, mais aussi la valeur de l'évaluation
indépendante, privée de l'information (\cite{page07};
\cite{landemore12}; \cite{lorenz11};
\cite{surowiecki04}). Les délibérations du groupe peuvent égalment
souffrir du respect des élites (du groupe)  ou de la solidarité de
groupe, conduisant soit à un groupe de pensées ou une
polarisation. (\cite{sunstein06})  
\item Enfin, il existe un conflit entre la maximisation du pouvoir
démocratique en permettant à un corps délibérant d'établir son propre
agenda, rédiger ses propres factures, et voter sur eux, contre
l'évitement de regroupement de questions, avec le vote-échange qui en
résulte, ainsi que des décisions arbitraires résultant de la force de
persuasion de quelques membres charismatiques non représentatifs
(\cite{sutherland08}). Ces cinq dilemmes (et résolutions
proposées) sont résumés dans le tableau 1 dans la section conclusion.
\end{enumerate}



