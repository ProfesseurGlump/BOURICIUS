\begin{titlepage}
  \hrulefill\par
  \vspace{0.05cm}
  \begin{Large}
      Démocratie par tirage au sort multi-corps : \par
      leçons athéniennes pour l'époque moderne.\par 
  \end{Large}
  \vspace{0.5cm}
  
  \begin{abstract}
    La démocratie athénienne classique est présentée (par l'auteur)
    comme un système représentatif, plutôt que sous une forme
    communément appelée «démocratie directe». Vu de cette façon, le
    problème d’échelle communément admis dans l’application de la
    démocratie athénienne aux états-nations modernes est résolu, et
    les principes pratiques du modèle athénien de la démocratie
    demeurent pertinents aujourd’hui. Le rôle clé du tirage au sort
    pour constituer de multiples corps délibérants est ici
    expliqué. Cinq dilemmes rencontrés dans les propositions modernes
    pour l’utilisation du tirage au sort sont examinés. Enfin, un
    nouveau modèle pour légiférer, utilisant plusieurs organismes
    désignés est présenté. Ce modèle résout les dilemmes et peut être
    mis en oeuvre de nombreuses façons : du petit ajout à un système
    existant jusqu'à une réforme plus fondamentale, comme le
    remplacement d’une ou deux chambres élues pour la législation.
  \end{abstract}

  \vspace{0.5cm}

  \begin{motcle}
    démocratie, tirage au sort, sélection aléatoire, innovation, demarchy
    (néologisme) , Athènes, représentation   
  \end{motcle}

  \vspace{0.5cm}

  \begin{merci}
    Je tiens à remercier John \textsc{Gastil}, Ethan \textsc{Leib} et
    spécialement David \textsc{Schecter} pour les informations utiles
    pour affiner cet article. 
  \end{merci}

  \vspace{0.5cm}

  % \begin{merci}[(du traducteur)]
  %   Je tiens à remercier l'auteur Terrill \textsc{Bouricius} pour cet
  %   article, qui je l'espère changera l'avenir du monde, le
  %   \emph{Journal of Public Deliberation} qui permet sa lecture et son
  %   téléchargement en ligne et gratuitement. Je tiens également à les
  %   remercier pour avoir accepté de me confier cette
  %   traduction. Enfin, je souhaiterais remercier David \textsc{Van
  %     Reybrouck} pour avoir écrit le livre \emph{Contre les élections}
  %   (actes sud février 2014) dans lequel il présente le travail de
  %   Terrill \textsc{Bouricius}; et je tiens à remercier mon père de me
  %   l'avoir prêté et ma mère pour son amour. 
  % \end{merci}


\end{titlepage}






