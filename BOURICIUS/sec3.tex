% !TEX encoding = UTF-8 Unicode

\section{Une proposition pour un système législatif basé sur le 
  tirage  au  sort} 
\subsection{Aperçu}
Cette proposition utilise une variété de corps, chacun avec des
caractéristiques uniques (tels que la méthode de sélection, et la
durée du mandat) qui sont optimisés pour la tâche de chaque corps. La
proposition ci-dessous est complète --- une sorte de "modèle de
référence" tel que celui utilisé par les architectes. Dans les
applications du monde réel, il est probable que certains organismes
décrits peuvent être utilisés, et d'autres pas. Par exemple, un
gouvernement de la ville peut utiliser le Conseil du Règlement, la
Commission d'Examen et la Jurys Politiques dans un domaine
politique spécifique, mais garder le Groupe d'Intérêt, le Conseil de
l'Agenda, et les fonctions du Conseil de Surveillance au sein de la 
ville existante.\par 
\subsection{Configuration de l'agenda (Conseil de l'Agenda et des pétitions)}
Un corps alloué appelé Agenda Conseil aurait la responsabilité de
fixer les ordres du jour des corps d'élaboration des politiques --
mais pour le développement de projets de loi, le vote sur eux, ou
toute autre chose. C'est ce que j'appelle un corps méta-législatif,
parce qu'il légifère sur la législation. Le système athénien n'a pas
complètement isolé l'établissement de l'élaboration de propositions
d'agenda, puisque le Conseil des 500 pouvait aussi jouer ce
rôle-là. Toutefois, dans les cas où l'Assemblée du peuple a entamé
l'examen de certaines nouvelles législation, la tâche de rédaction
incomberait aux citoyens, avec la décision ultime revenant à un comité
législatif imparti. L'affectation des tâches de méta-législatif, à un
organisme distinct de la législtation normale suit le principe de
longue date de "checks and balances", ou la séparation des pouvoirs
tel que préconisé par les goûts de \textsc{Montesquieu} et
\textsc{Madison}.\par
Cet organisme peut être sélectionné en utilisant un système de loterie
de volonté à deux vitesses, similaire à celle utilisée dans
l'Assemblée Britannique citoyens britanniques en 2003-4 (\cite{herath07:_real_power_peopl}). Un tel système de loterie à deux vitesses a également été
utilisé à Athènes, où un groupe de 6 000 citoyens âgés de plus de 30
ans étaient choisis pour un mandat d'un an pour servir dans les
tribunaux des personnes et des groupes législatifs, avec un
sous-ensemble sélectionné par tirage au sort pour tout procès
donné ou de droit.\par
Si plusieurs facteurs démographiques sont équilibrés en fonction de la
diversité de la population générale, la sélection aléatoire aura
également tendance à produire un corps qui ressemble étroitement à la
population générale en termes d'autres caractéristiques, comme les
attitudes politiques et les styles cognitifs.\par
Le Conseil de l'agenda et son personnel sont en quête de
problèmes nécessitant une attention particulière, plutôt que de
simplement réagir aux médias ou à des pressions de groupes d'intérêt
particulier. Par exemple, les USA font face maintenant à un peu discuté,
mais incontestable, déficit d'infrastructure (transport, réseaux
d'eau, etc) qui sans doute est ignoré par les élus parce que soulever
la question ne garantie pas de réélection. L'objectif est d'établir
un programme rationnel, plutôt que selon les préceptes d'impératifs
électoraux.\par
Dans l'esprit de \emph{isegoria}, il serait également souhaitable de disposer
d'un autre moyen pour l'établissement du programme, ouvert à tous les
citoyens. Par conséquent, personne ne serait autorisé à lancer une
pétition pour forcer un sujet sur l'agenda. Etablir des règles
qui permettraient à tout citoyen de promouvoir l'agenda, mais
qui n'encourageraient pas les intérêts particuliers à inonder l'ordre du
jour, est un défi. Le seuil et les règles pour un tel effort de
pétition ne devraient pas être pris par le Conseil de l'agenda,
car il pourrait être tenté de défendre ses prérogatives en établissant
des barrières déraisonnables aux pétitions. Au lieu de cela, un
organisme distinct alloué appelé Conseil du Règlement (voir
ci-dessous) s'acquitterait de cette tâche, d'adapter les règles au fil
du temps en cherchant à optimiser \emph{isegoria}, tout en évitant à un intérêt
particulier de dominer.\par
\subsection{\'Elaboration des projets (groupes d'intérêt)}
Une fois l'agenda est établi, il y aurait un appel aux
volontaires pour servir dans un groupe d'intérêt, un groupe d'une
dizaine de membres (pour faciliter la participation active). Les
groupes d'intérêt génèreraient des propositions législatives, mais
n'auraient pas le pouvoir de les adopter. A Athènes, les citoyens
eux-mêmes sélectionnés (ho boulomenos) pouvaient proposer des lois ou
des décrets, mais ceux-ci devaient généralement passer par de multiples
organismes auto-sélectionnés et alloués (l'Assemblée, le Conseil des 500
et le groupe législatif) avant l'adoption finale.\par
Il y auvait autant de groupes d'intérêt sur un sujet donné
que le nombre de bénévoles permettait d'en créer. Cela dérivait
du, mais aussi modifiait le, principe de \emph{isegoria}. Contrairement à
Athènes, dans ce cas, la personne ne parle pas directement au corps
de décision ultime. Cependant, la division en plusieurs petites unités
désigne le montant de l'entrée qui serait beaucoup plus élevé, avec la
possibilité pour tout participant à un comité d'affecter la
législation finale. Les groupes d'intérêt au niveau local 
pourraient se rencontrer en personne le week-end ou le soir, mais, comme
l'accès à Internet devient de plus en plus commun dans toute la
société, beaucoup seraient susceptibles d'utiliser des outils de
collaboration sur Internet qui permettent aux membres de communiquer
et de travailler sur les propositions de manière séquentielle, plutôt
que de devoir coordonner des réunions.\par
Les groupes d'intérêt pourraient être formés à plus d'un titre.
Certains groupes d'intérêt pourraient être auto-organisés par  
des personnes partageant les mêmes idées. Cela pourrait conduire à la
génération de propositions extrêmes. Cependant, comme ces groupes
sauraient qu'ils ne sont pas les décideurs finaux, ils auraient intérêt
à tempérer leurs propositions afin de gagner l'approbation de la phase
finale. Alternativement, les bénévoles peuvent être mélangés de façon
aléatoire sur les groupes d'intérêt afin de promouvoir la diversité des
perspectives et des styles cognitifs. \par
L'auto-sélection au niveau du groupe d'intérêt permet aux experts qui
seraient inéligible (en raison de leur apparence, classe, la
personnalité, ou d'autres raisons) de contribuer à la gouvernance. Cela
signifie également que les intérêts particuliers et les incapables
auto-trompés pourraient participer. Même si un groupe d'une douzaine
serait susceptible d'identifier et d'écarter les idées de "tarés", il
est probable que certains groupes d'intérêt mettraient en avant des
propositions législatives pauvres. C'est une des raisons pour
lesquelles il est souhaitable d'avoir plusieurs groupes d'intérêt, et
pourquoi les groupes d'intérêt auto-sélectionnés ne doivent pas prendre
de décisions ultimes. Certains d'entre eux pourraient faire l'impasse,
ou se désintégrer en raison de défaillances de quorum. Dans la plupart
des cas, cependant, plusieurs groupes d'intérêt produiraient des
projets de propositions pour le prochain niveau d'corps désigné --- la
Commission d'examen.\par
\subsection{Examen des projets de lois (commissions)}
Il y aurait une commission d'examen unique pour chaque domaine
d'action établi par le Conseil des règles. Le Conseil des 500 est
l'analogue le plus proche d'Athènes. Les comités d'examen remplissent
la plupart des fonctions d'un corps législatif traditionnel, sauf
l'initiation et l'adoption finale de la législation. Le processus de
la commission d'examen serait très différent de celle d'un parlement
élu. Le Conseil du règlement pourrait établir un processus proche de
celui de l'Assemblée des citoyens de Colombie-Britannique, avec une
"phase d'apprentissage", une "phase de délibération", etc. \par
Bien que la compétence de chaque commission d'examen est beaucoup plus
large que les groupes d'intérêt, elle est beaucoup plus étroite que les
assemblées législatives existantes, ou les modèles de deuxième chambre
de tirage au sort proposés, qui traitent de toutes les questions. Cela
permet aux membres de développer une compréhension plus profonde dans
une zone définie qu'il n'est possible dans une assemblée législative qui
englobe tout. Dans les assemblées législatives traditionnelles, il n'y a
généralement qu'une petite partie des membres --- ceux qui servent pour
un comité particulier --- qui ont une chance de comprendre pleinement
chaque projet de loi. La plupart des législateurs ne lisent même pas
la majeure partie des projets de loi qu'ils votent. Ils votent
inévitablement sur la base d'autres considérations, telles que le vote
échange ("Je vais voter pour votre route , si vous votez pour ma
modification de subvention de l'éducation"), ou sur certains 
heuristique - généralement à l'instar de leurs camarades de parti qui
servent pour un comité de projet de loi de référence. Ainsi,
l'homogénéité de la législature dans son ensemble est aggravée en
ayant seulement une petite fraction de l'ensemble de corps même en
tentant de comprendre chaque projet de loi. Le concept de commission
d'examen encourage chaque membre cherchant à comprendre chaque projet
de loi, et élimine ou réduit au moins, le vote-échange et les jeux
partisans.\par
Les comités d'examen seraient choisis de la même manière que le Conseil
de l'agenda --- une loterie de la volonté. Contrairement  aux
groupes d'intérêt, cependant, des volontaires pour la loterie de la
commission d'examen ne choisiraient pas le sujet auquel ils seraient
affilié, afin d'éviter toute distorsion d'intérêt particulier. \par
Les comités d'examen seraient plus importants (en nombre) que les
groupes d'intérêt (peut-être 150 membres au niveau de l'Etat). Ils
seraient raisonnablement rémunérés, et fournis en repas, garde
d'enfants, et un environnement de travail agréable. \par
Pour un état ou un organisme national, ces groupes seraient à
temps plein, avec des termes de chevauchement de peut-être trois ans (pour se
familiariser avec le sujet). Pour une mise en oeuvre municipale, il
pourrait être approprié de tenir des réunions le week-end ou le soir
afin de ne pas interférer avec l'emploi normal. Cet organisme serait
beaucoup plus une description représentative que les groupes
d'intérêt. La projection de membres potentiels devraient être minimes
(telles que la capacité à lire et comprendre des documents
d'information) afin de ne pas fausser indûment la représentativité du
corps final. \par
La commission d'examen exercerait des activités législatives
traditionnelles: la tenue d'audiences, en invitant et en écoutant les
témoins experts, en utilisant le personnel professionnel pour la
recherche et la rédaction, et modifiant ou en combinant des éléments
des propositions soumises par les groupes d'intérêt afin de produire un
projet de loi final. Les Commissions d'examen peuvent également fixer
des objectifs ou des critères de factures définitives et reporter les
projets de retour au groupe d'intérêt pour la révision.\par
Les procédures utilisées par le Groupe d'étude exigent une conception
soignée de manière à maximiser le problème potentiel de résolution et
de réduire à la fois la pensée de groupe et la polarisation interne
(\cite{sunstein05:_why_societ_need_dissen}). Ces tendances
psychologiques sont puissantes, mais peuvent être traitées avec un bon
design (\cite{manin05:_democ_delib}; \cite{fishkin09:_when_peopl_speak}). Les éléments
nécessaires aux niveaux micro et macro pour une véritable
"délibération démocratique" méritent une étude intense 
(\cite{gastil12:_makin_direc_democ_delib_random_assem}). Ces éléments de conception pourraient inclure
l'élaboration d'un ensemble de faits convenus comme une base de
discussion. Trop souvent, les membres des groupes parlent simplement
de leur passé parce que chacun a sa propre compréhension distincte des
"faits". L'alternance entre orateur pour et contre pour tout
amendement est préférable à un style de discussion où la persuasion de
la majorité initiale domine. Les psychologues ont montré que les gens
ont tendance à pencher vers le côté de la majorité apparente
simplement par désir de s'intégrer dans --- une sorte de lien
social instinctif. Cela peut également conduire à une polarisation en
sous-groupes qui s'éloignent de plus en plus que les membres adhèrent
aux arguments qui soutiennent leur position initiale, et de rejeter
les arguments qui ne soutiennent pas leur point de vue 
(\cite{sunstein05:_why_societ_need_dissen}). Pourtant, la communication entre les membres du groupe
comprenant des informations à propos des niveaux de confiance des membres a
montré une amélioration des décisions dans de nombreuses situations
(\cite{koriat12:_when_are_two_heads_better}). Comme la science groupe de prise de décision avance,
les procédures des institutions démocratiques doivent être ajustés en
conséquence.\par
\subsection{Vote sur les projets de lois (jurys politiques)}
Un élément clé de cette proposition est que nous ne laissons pas les
décisions finales sur la politique à un groupe d'intérêt ni à une
commission d'examen. En raison du risque de pensée de groupe, ou d'une
extrême polarisation d'une proposition soutenue par la majorité, les
décisions finales sont prises par des corps distincts, appelé jurys
politique. Cette séparation permettrait également de réduire la
probabilité de positions majoritaires extrêmement polarisées sortant
de la comission d'examen, puisque les membres des groupes d'intérêt
particuliers et commission d'examen devraient comprendre que les produits
finaux doivent être en mesure de réussir l'examen devant le jury
politique. Trouver un terrain d'entente et répondre aux besoins des
perspectives minoritaires pourraient améliorer leurs chances de succès
final.\par
Chaque jury politique votera sur un projet de loi, comme les groupes
législatifs d'Athènes. Dans un état ou la mise en oeuvre nationale, le
service de jury serait théoriquement obligatoire, mais avec des
excuses de difficultés raisonnables. Il est peu probable qu'une mise
en oeuvre rapide au niveau municipal aurait le pouvoir juridique de
mandater le service, donc des moyens d'encourager la participation
devraient être examinés. Nous avons peu ou pas de possibilités
comparables pour l'engagement civique dans l'Amérique moderne, si on
ne peut que spéculer sur les taux de participation potentiels. Durées
de service courtes associées à la rémunération adéquate, et donc la
représentativité descriptive. Un état ou le jury  de la politique
nationale devrait probablement avoir au moins 400 membres pour
obtenir un échantillon représentatif. Un jury de la politique
municipale serait probablement nettement plus petit, tout simplement
pour des raisons financières.\par
Comme les membres du jury de la politique, comme les groupes
législatifs et les tribunaux populaires d'Athènes, sont tout
simplement à l'écoute de présentations, sans discussion, il serait
logistiquement possible de créer des jurys beaucoup plus grand --- des
milliers --- grâce à l'utilisation de l'Internet. Cependant, une raison
impérieuse pour une taille plus modeste est le but de trouver le
"sweet spot"(bon endroit--- on dirait plutôt juste milieu)  qui motive
les participants à surmonter l'ignorance rationnelle de l'électeur et
de fournir les efforts nécessaires pour porter un jugement
considéré.\par
La tâche et les procédures du jury politique sont fondamentalement
différentes de celles des groupes d'intérêt et avis. Comme les groupes
législatifs et les tribunaux populaires d'Athènes, un jury politique
assisterait à des présentations pour et contre le projet de loi, et
sans débat plus approfondis, voterait à bulletin secret. Cette
procédure destinée à profiter de la sagesse des foules décrites dans
le livre de James Surowiecki tout en évitant la dynamique de pensée de
groupe et de polarisation qui peuvent survenir lorsque les
participants s'engagent dans la discussion (\cite{surowiecki04:_wisdom_crowd}). Le
rétrécissement de la tâche à l'écoute de présentations et de vote, et
l'incitation de l'exercice du pouvoir réel, augmentera la volonté de
participer et de représentativité ainsi descriptive.\par
Le scrutin secret permet également d'éviter les pressions sociales qui
peuvent interférer avec la capacité de voter comme l'entendent les
membres. Le vote secret réduit également le risque d'achat de
voix. Les protections d'altération du jury seraient encore
nécessaires, mais les corrupteurs potentiels sont moins tentés
d'essayer d'acheter un vote quand il n'y a pas moyen de savoir si le
vote a été livré.\par
Les membres du jury de la politique ne sont pas destinés à
"représenter" les constituants géographiques (comme dans un système
électoral), ni les constituants particuliers (par exemple basés sur la
démographie ou les opinions politiques). Le concept traditionnel de la
responsabilité des représentants ne s'applique tout simplement
pas. Cela peut sembler étrange au premier abord, mais une fois que
l'analogie du jury est bien comprise, cela devient évident. Nous nous
attendons à ce que des jurés d'une race particulière, par exemple,
demande justice, plutôt que d'être "responsable" pour les citoyens de
leur race. Chaque membre d'un jury politique est invité à voter pour
ce qu'il pense être le meilleur, avec le résultat net en miroir de ce
que la communauté dans son ensemble peut décider si elle disposait
de l'information et du temps pour réfléchir. L'expérience avec les
sondages délibératifs ménée par James \textsc{Fishkin}, dans laquelle les
membres de la communauté choisis au hasard sont invités à prendre des
décisions sur des questions de politique publique, suggèrent que ces
groupes représentatifs peuvent être plus en mesure de favoriser à long
terme et les intérêts communautaires sur l'intérêt égocentrique et les
représentants élus (\cite{fishkin09:_when_peopl_speak}). Même si aucun élu n'était corrompu
ou égoïste, la dynamique de réélection peut les pousser à prendre des
décisions "populaires" qu'ils croient que leur, circonscription myope ou
égoïste mal informée favoriserait, plutôt que des décisions qui
pourraient être véritablement dans leur intérêt à long terme de
constituants.\par
Pour améliorer cette promotion du bien commun, la recherche sur les
effets d'amorçage (\cite{aquino09:_testin_social_cognit_model_moral_behav}) suggère qu'il pourrait également être
bénéfique d'avoir des membres qui en font une promesse, similaire au
serment d'Athènes Heliastic, qu'ils voteront de façon impartiale selon
ce que leur conscience leur suggère qui soit le plus juste et le
meilleur pour la communauté. Mais même si nous supposons que la
plupart des membres d'un corps choisi au hasard ignorent le bien
commun (ou l'on suppose que la notion même de bien commun est un vain
mythe), et au lieu de voter selon les intérêts égoïstes, au pire, nous
arrivons au résultat idéal envisagé par la démocratie libérale
contradictoire, qui est de trouver la préférence de la majorité des
intérêts concurrents.\par
\subsection{\'Elaboration des règles de la procédure (Conseil des règles)}
Le comité des règles de la législature traditionnellement élu a un
conflit intégré d'intérêt, où la partie qui contrôle adapte les règles
ou leur interprétation, à favoriser le parti au pouvoir et l'atteinte
d'objectifs législatifs. Pour résoudre ce problème, je propose la
création d'un Conseil des règles imparties, de forme semblable au
Conseil de l'agenda. Le Conseil du Règlement établirait des
règles et des procédures pour tous les autres groupes et des conseils,
tels que le processus de loterie, les exigences de quorum, des moyens
pour solliciter des témoignages d'experts, les procédures qui seront
utilisés dans la délibération, etc les membres auraient des durées
limitées et ne pourraient pas connaître la façon dont les règles
pourraient blesser ou aider n'importe quel morceau particulier de la
législation future. Leur intérêt naturel serait d'assurer le plus
beau et le meilleur fonctionnement de tous les corps. Il pourrait
être opportun de limiter la loterie pour le Conseil particulier à ceux
qui ont déjà servi sur un autre corps désigné, afin qu'ils
comprennent les dynamiques à l'oeuvre. \par
\subsection{Application des règles (Conseil de surveillance)}
Qui présente les arguments pour et contre à des jurys politique, et
comment décident-ils de faire exactement ce qu'inclut le contenu, dans
le but de donner une présentation "équilibrée" ? Même le charisme
relatif, l'apparence ou le statut social des différents intervenants
peuvent être importants. Une approche possible est d'avoir le même
personnel présent à la fois pro et arguments anti fabriqués par les
membres des groupes d'intérêt et d'examenn, qui tombent de chaque côté
de la fracture pro anti. Toutefois, cela laisse le jury de la
politique en danger d'être piloté par la bureaucratie de promouvoir
ses propres intérêts --- un problème commun avec les corps élus. 
Beaucoup de législatures d'Etat et nationaux tentent de résoudre
ce problème en permettant à chaque législateur d'engager leur propre
personnel. Toutefois, ce personnel individualisé finit souvent par
passer une quantité excessive de temps sur des préoccupation de la
réélection, tels que les relations publiques et le service des
constituants, plutôt que sur la politique.\par
Au lieu de cela, je propose la création d'un \emph{Conseil de
  surveillance}, désigné par le sort, qui porte exclusivement sur la
performance du personnel et de l'équité, plutôt que les questions de
politique elles-mêmes. En plus d'évaluer la performance générale de
l'équipe, ils se prononcent sur les plaintes concernant des présentations
biaisées ou injustes données par le personnel. Ils doivent
probablement avoir le pouvoir d'embaucher et de licencier du personnel
au service des autres corps de tirage au sort.\par
Au niveau national ou régional, il pourrait être approprié d'avoir
trois conseils méta-législatifs distincts Conseil Agenda fixant
l'agenda des questions à traiter et établir des groupes
d'intérêt, le Conseil du règlement établissant les règles et
procédures du bon déroulement de délibération, et le Conseil de
contrôle qui supervise le personnel de tous les corps issus du
tirage au sort, et peut-être celui qui supervise la performance de
l'exécutif dans l'application des lois. Pour une municipalité, un
conseil unique méta-législatif semble approprié; la tâche limitée
d'assurer que le processus de constitution des coprs par tirage
au sort soit bon et le personnel qui fournissent des présentations
équilibrées.\par









  















