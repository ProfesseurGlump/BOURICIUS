%% raccourcis pour faire des étiquettes (label)


% pour aligner du texte

% à gauche 
\newcommand{\ag}[1]{{\flushleft #1}}

% à droite 
\newcommand{\ad}[1]{{\flushright #1}}


% pour les couleurs

\newcommand{\blue}[1]{\textcolor{blue}{#1}}
\newcommand{\brown}[1]{\textcolor{brown}{#1}}
\newcommand{\cyan}[1]{\textcolor{cyan}{#1}}
\newcommand{\green}[1]{\textcolor{green}{#1}}
\newcommand{\lime}[1]{\textcolor{lime}{#1}}
\newcommand{\olive}[1]{\textcolor{olive}{#1}}
\newcommand{\orange}[1]{\textcolor{orange}{#1}}
\newcommand{\pink}[1]{\textcolor{pink}{#1}}
\newcommand{\purple}[1]{\textcolor{purple}{#1}}
\newcommand{\red}[1]{\textcolor{red}{#1}}
\newcommand{\teal}[1]{\textcolor{teal}{#1}}
\newcommand{\yellow}[1]{\textcolor{yellow}{#1}}


\definecolor{magenta}{RGB}{255,0,255}
\newcommand{\magenta}[1]{\textcolor{magenta}{#1}}

\definecolor{violet}{RGB}{127,0,255}
\newcommand{\violet}[1]{\textcolor{violet}{#1}}

\definecolor{orang}{RGB}{255,127,0}
\newcommand{\orang}[1]{\textcolor{orang}{#1}}

% pour une couleur et un soulignement
\newcommand{\cu}[1]{\cyan{\underline{#1}}}

% pour définir une liste de mots-clés avec la présentation comme abstract
\newenvironment{motcle}
        {\begin{center}\normalfont\bfseries Mots-clés
        \end{center}\begin{quote}}{\end{quote}\par}
 
% pour définir les remerciements avec la présentation comme abstract
\newenvironment{merci}[1][(de l'auteur)]
        {\begin{center}\normalfont{\bfseries Remerciements} #1
        \end{center}\begin{quote}}{\end{quote}\par}


% pour définir une recommandation avec la présentation comme abstract
\newenvironment{recomd}
        {\begin{center}\normalfont\bfseries Citation recommandée
        \end{center}\begin{quote}}{\end{quote}\par}
