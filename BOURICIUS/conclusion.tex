% !TEX encoding = UTF-8 Unicode

\section{Conclusion}
Les principes et les pratiques de la démocratie athénienne classique
constituent un point de départ pour la conception d'une démocratie
moderne, qui peut être adapté à n'importe quelle taille. Le modèle
présenté ici répond aux principes de l'égalité politique des Athéniens
(isonomia) et la liberté de s'exprimer et de contribuer au processus
de prise de décision démocratique (isegoria). Il capitalise sur la
pratique athénienne par tirage au sort, en fournissant un système de
freins et de contrepoids. Enfin, ce modèle permet de résoudre les
dilemmes des cinq objectifs contradictoires, évoqués ci-dessus (voir le
tableau \ref{tab1}).

\begin{table}[m]
  \caption{Résolution des dilemmes par utilisation de multiples
corps tirés au sort}\vspace{0.2cm}
  \begin{tabular}{*{3}{|c}|}
    % {|>{\raggedleft\hspace{0pt}}p{15mm}%
    %  |>{\centering\hspace{0pt}}p{15mm}%
    %  |>{\raggedright\hspace{0pt}}p{15mm}|}      
    \hline % LIGNE 1
    \textbf{Premier objectif}
    &
    \textbf{Objectif en conflit}
    &
    \textbf{Résolution}
    \\  
    \hline % LIGNE 2
    \begin{minipage}[c]{.33\linewidth}
          maximiser la représentativité descriptive
    \end{minipage}
    &
    \begin{minipage}[c]{.33\linewidth}
      maximiser l'intérêt et l'engagement des membres d'un corps
      délibérant
    \end{minipage}
    &
    \begin{minipage}[c]{.33\linewidth}
      les jurys politique maximisent la représentativité descriptive,
      tandis que les conseils et les groupes assurent l'engagement
    \end{minipage} 
    \\ 
    \hline % LIGNE 3
    \begin{minipage}[c]{.33\linewidth}
      rotation fréquente, élargir la participation et augmenter la
      résistance à la corruption
    \end{minipage}
    &
    \begin{minipage}[c]{.33\linewidth}
      mandats plus longs, afin de 
      maximiser la familiarité des participants avec les questions à
      l'étude
    \end{minipage}
    &
    \begin{minipage}[c]{.33\linewidth}
      les jurys politique ont une rotation fréquente et une large
      participation, tandis que les conseils et les groupes permettent
      le développement des compétences
    \end{minipage} 
    \\
    \hline % LIGNE 4
    \begin{minipage}[c]{.33\linewidth}
      droit garanti à chaque citoyen de participer
    \end{minipage}
    &
    \begin{minipage}[c]{.33\linewidth}
      éviter la domination par des intérêts particuliers en raison de 
      l'auto-sélection
    \end{minipage}
    &
    \begin{minipage}[c]{.33\linewidth}
      groupes d'intérêt et pétitions permettent à
      tout citoyen de contribuer, pendant que les Comissions d'examen,
      et, finalement les jurys politique protègent contre la
      distorsion de la participation
    \end{minipage} 
    \\
    \hline % LIGNE 5
    \begin{minipage}[c]{.33\linewidth}
      maximiser la capacité de résolution de problèmes par la discussion
      et le débat interne
    \end{minipage}
    &
    \begin{minipage}[c]{.33\linewidth}
      éviter la polarisation et la pensée de
      groupe
    \end{minipage}
    &
    \begin{minipage}[c]{.33\linewidth}
      Conseils et groupes promeuvent et prennent la résolution de
      problèmes alors que les jurys politique protègent à la fois contre
      la polarisation et la pensée de groupe
    \end{minipage} 
    \\
    \hline % LIGNE 6
    \begin{minipage}[c]{.33\linewidth}
      maximiser le pouvoir démocratique en ayant un corps délibérant de
      larges pouvoirs fixer son propre agenda, les projets de ses
      propres projets de loi et voter le haut ou vers le bas 
    \end{minipage}
    &
    \begin{minipage}[c]{.33\linewidth}
      éviter les décisions arbitraires qui sont trop sensibles aux
      inclinations 
      de quelques membres charismatiques non représentatifs dans un
      corps alloué 
    \end{minipage}
    &
    \begin{minipage}[c]{.33\linewidth}
      les Conseils permettent un contrôle démocratique de l'agenda
      alors que les jurys protègent contre l'influence des extrêmes et
      individus puissants 
    \end{minipage} 
    \\
    \hline
  \end{tabular}
  \label{tab1}
\end{table}

Ce modèle de tirage au sort est évolutif à partir d'une municipalité à
l'autre, national ou international. Les domaines de compétence sont
flexibles sans fin, des décisions relativement mineures jusqu'à et y
compris les tâches législatives, à l'exclusion des
élections. Evidement, il est à la fois pratique et prudent de
commencer à une petite échelle, et avec un mandat limité. Ainsi, un
accent initial sur les adoptions municipales de sens, comme cela s'est
proudit avec le Canada Bay en Nouvelle-Galles du Sud, en Australie en
2012 (\cite{thompson12}). Situations où les élus actuels sont
réticents à prendre des décisions politiques 
impopulaires (politique no-win), et pourraient être disposés à les
transférer à des corps de citoyens désignés, sont particulièrement
plausible pour l'adoption anticipée.   \par
Ce modèle peut être mis en oeuvre dans une variété de façons, allant
de petits changements progressifs à des réformes fondamentales. Par
exemple, les éléments du modèle pourraient être utilisés pour : 
\begin{enumerate}
\item Face à une loi --- comme avec l'Assemblée de citoyens de
  Colombie-Britannique 
\item Faire toutes les lois au sein d'une zone d'émission --- par
  exemple, une zone où les législateurs ont un conflit d'intérêt, tels
  que le redécoupage ou application de règles d'éthique  
\item Améliorer la qualité de délibération d'une initiative et le
processus de référendum (\cite{gast12}) --- par exemple, que dans le
processus Initiative Oregon avis.
\item Remplacer une chambre élue par un parlement bicaméral
\item Effectuer l'ensemble du processus législatif à la place d'une
assemblée législative élue.
\end{enumerate}
C'est mon espoir que cette "conception de référence" puisse servir de
base conceptuelle pour l'élaboration de propositions concrètes, que
peuvent retenir une chambre législative élue et soit donc plus facile
à mettre en oeuvre, mais qui bénéficient des nombreux avantages de ce
modèle. On se demande ce qui aurait lieu si cette vision concurrente
de la démocratie avait été largement connue par les Egyptiens pendant
le printemps arabe. L'ironie du militant démocratique
\cite{ziada12} demandant un boycott d'une élection, en raison de 
l'absence d'un candidat accpetable (\cite{ziada12}), conduit à la
maison les limites du modèle de démocratie électorale. La démocratie
peut-être mieux sans élections.\par

